\subsection{Classical and quantum computing integration}

In the literature, it is possible to find different ways to use quantum computing \cite{Elsharkawy2023}. The first approach is \emph{standalone}, in which classical computing is not considered to solve the problem. Instead, quantum problems are solved only with quantum computers. The second option is \emph{co-located integration}, where a classical system interacts with one or more quantum computers via network calls. Lastly, it is possible to do an \emph{on-node integration} that combines classical and quantum elements on the same chip.

Currently, "standalone" is not technologically ready to solve real problems. The literature has reported various restrictions on the pure quantum phenomena coined as the non-cloning theorem \cite{Wootters_Zurek_1982}, the non-communication theorem \cite{Popescu_Rohrlich_1998}, the no-teleportation theorem \cite{Pathak_2013} and the no-deleting theorem \cite{Kumar_Pati_Braunstein_2000}. Consequently, this survey aims to investigate quality attributes that matter in the hybrid approach, including co-located and on-node integration.

%Currently, quantum hardware requires layers of classical systems, so "stand-alone integrations" are not technologically ready to solve real problems. The limitations to solving problems using pure quantum, including restrictions on the phenomena reported as the non-cloning theorem \cite{Wootters_Zurek_1982}, the non-communication theorem \cite{Popescu_Rohrlich_1998}, the no-teleportation theorem \cite{Pathak_2013} and the no-deleting theorem \cite{Kumar_Pati_Braunstein_2000}. Consequently, this survey aims to find quality attributes of hybrid systems.

\subsection{Classical software attributes and metrics}

As a first step to understand quality in quantum software, it is useful to try to extrapolate concepts well-known in traditional software development. Every modern software development project faces the challenge of defining quality. As described in the ISO\slash IEC 25002: 2024 \footnote{Systems and software Quality Requirements and Evaluation (SQuaRE), https://www.iso.org/standard/78175.html}, quality includes a wide variety of characteristics. % that can be categorized into one of the following groups: functional suitability, performance efficiency, compatibility, usability, reliability, security, maintainability, and portability.
In 2023, a survey named \textit{"Quality Metrics in Software Architecture "}\cite{Silva2023} presented a report on the state of the art in metrics for evaluating quality in software. This survey highlighted the most important attributes to evaluate the quality of classical software, such as functional suitability (distance, suitability), performance efficiency (response time, throughput), compatibility (interoperability), usability (coupling/decoupling, number of runnables and Inter-runnable per software, understandability), reliability (availability, mean time between failures, reliability), security (cohesion, complexity, inheritance, polymorphism), maintainability (best practices adherence, decision traceability, decomposition), and portability (dependency). These attributes provide a comprehensive framework for assessing software quality.

According to \cite{Gill2022}, evaluating a quantum solution with traditional attributes/metrics using a black-box strategy is possible, despite the different physical principles governing these systems. This approach underscores the relevance of classical software metrics in the quantum domain, while also highlighting the necessity to explore attributes and metrics specific to quantum systems. Such exploration will help developers create algorithms, gates, and circuits specifically for the quantum world.

\subsection{Quantum software attributes and metrics}

The Talavera Manifesto \cite{Piattini2020}, created by a group of quantum development experts, has established principles and commitments for designing quantum software. While it has provided a foundational framework, it has not included specific attributes or metrics, thus revealing a gap within the current understanding of quantum software quality.

In tandem, Cruz-Lemus' work \cite{26} has offered a review of existing quality attributes and metrics, analyzing them and proposing a set of metrics specifically for quantum circuits. This research has underscored the importance of developing tailored metrics for quantum systems, although it also pointed out the need for empirical testing to validate these metrics.

Further emphasizing the need for dedicated research, Serranos' work \cite{Serrano2022} provides a general overview of key aspects of quality in quantum systems, such as compatibility and reliability. However, it affirmed that "\textit{a lot of work is still needed to achieve adequate quality levels in quantum software platforms}," setting the stage for future exploration and development in this field.

