Quantum computing is an emerging field based on the principles of quantum mechanics \cite{Maslov2019} enabling systems to leverage phenomena such as superposition and entanglement to address computational challenges that are infeasible for classical systems. This novel approach holds the potential to improve the way humans solve complex problems such as large-scale number factorization \cite{Willsch2023}, cryptography \cite{Upadhyay2019}, model training \cite{Biamonte2017, AdvancesinQuantumDeepLearning}, and any problem that requires high computational resources. %Despite its promise, quantum computing is still in its nascent stages, with significant challenges to overcome before it can be widely adopted.

% es tendencia
According to Gartner and McKinsey reports \cite{McKinsey}, recent technological advances and increased funding have driven strong momentum in the field. Notwithstanding, quantum computing is still in its nascent stages, with significant challenges to overcome before it can be widely adopted, particularly in integrating quantum capabilities into classical computing environments—a critical step for realizing its full potential. %This work investigates some of these challenges from a software architecture perspective, focusing on quality attributes and integration issues.
%\footnote{https://biforesight.com/quantum/quantum-market-forecast-no-hype-mckinsey-and-globaldata-have-seen-the-future/}

% hay avances pero hay mucho por hacer en cuanto a la integración
Over the past decade, quantum software engineers have developed domain-specific programming languages such as Quantum Computation Language (QCL) \cite{6} and Q\# \cite{7}, alongside software development kits (SDKs) like Azure Quantum \cite{8}, Ocean SDK \cite{9}, and Qiskit \cite{10}. However, while these tools represent significant progress, their integration into enterprise product lifecycles is still in its early stages \cite{Ruefenacht2022}. Recent efforts, such as the creation of an enterprise service bus to connect classical and quantum systems \cite{Bonilla2024}, highlight the ongoing need for robust integration frameworks.

%Quantum hardware complexity further complicates this integration. Different physical phenomena—such as trapped-ion qubits \cite{Blinov2004}, superconducting qubits \cite{14}, and photonic qubits \cite{Niemietz2021}—require diverse architectures and protocols, developed by companies like D-Wave \footnote{D-Wave Systems, https://www.dwavesys.com/}, IBM \footnote{IBM Quantum Platform, https://quantum.ibm.com/}, and Google \footnote{Google Quantum AI, https://quantumai.google/}. This diversity underscores the difficulty of defining standardized quality attributes for quantum services, while emphasizing the challenge of building resilient systems that seamlessly integrate quantum and classical technologies.

The purpose of this study is to examine hybrid systems from a software architecture perspective, focusing on quality attributes. By \emph{hybrid} we mean the integration of classical computing resources with quantum hardware \cite{Sodhi2021}. The classical layers include the software application written in a high-level programming language, the network and scaffolding code (libraries, toolkits, domain specific languages, etc.) to interact with quantum devices. %A detailed explanation of the hybrid computing system is presented by Balwinder in \cite{Sodhi2021}, under what is referred to as an "architecture of quantum computing platform".

Hybrid computing systems present several challenges, which include: (i) the heterogeneity of quantum hardware due to multiple quantum phenomena (e.g., trapped-ion/superconducting/protonic qubits) \cite{Elsharkawy2023} \cite{Sodhi2021}, (ii) the multiple ways to represent data in Qubits \cite{Moguel2022}, (iii) the lack of software standardization \cite{Elsharkawy2023} \cite{Cerezo2022} \cite{Sodhi2021}, (iv) the lack of monitoring/observability tools \cite{Moguel2022} \cite{40}, (v) Error-prone and complexity of developing algorithms using quantum gates (low-level abstractions) \cite{Sodhi2021}, (vi) the shortage of quantum developer experts \cite{Sodhi2021}, and (vii) the limited visibility on the boundaries between quantum and classical computing \cite{Vietz2021}.

Additionally, El Aoun et al. \cite{el_aoun2021understanding} summarize the challenges faced by quantum software engineers, including interpreting quantum program outputs, understanding quantum principles in code, and managing quantum-classical integration. They highlight the need for improvements in quantum ecosystem software development tools, such as error-handling, testing libraries, and library standardization.

% Indeed, the lack of standardization is somehow related to the heterogeneity of quantum hardware offered by industry partners today. Superconducting qubits are available from companies like Oxford Quantum Circuits (OQC) \footnote{OQC: https://oqc.tech/} , IQMs \footnote{IQMs: https://www.meetiqm.com/}, and Rigetti \footnote{Rigetti: https://www.rigetti.com/}. Similarly, neutron atom qubits can be accessed through QuEra \footnote{QuEra: https://www.quera.com/}, trapped ion technology is available on IonQ \footnote{IonQ: https://ionq.com/}, and photonic quantum computers are offered in Xanadu \footnote{Xanadu: https://www.xanadu.ai/}. Since all of those machines use different quantum phenomena for computing, and there is no standard way to create the quantum code, a variety of frameworks are constantly appearing/evolving with the goal of creating a unified way of interacting with the computers; among them we can list: Amazon Brakets \footnote{Amazon Brakets: https://aws.amazon.com/braket/}, qBraid \footnote{qBraid: https://www.qbraid.com/}, Pennylane \footnote{Pennylane: https://pennylane.ai/}, and Pytket \footnote{Pytket: https://docs.quantinuum.com/tket/}.

These challenges and advancements motivate this systematic mapping study, which examines progress in software quality for hybrid quantum-classical systems. Current literature often focuses on specific topics such as quantum verification \cite{22}, compilation \cite{Campbell2018}, simulation \cite{25}, and development \cite{26}. However, these studies typically adopt a theoretical perspective without addressing practical strategies for software quality attributes. %Moreover, the nascent state of quantum development tools \cite{27} makes this field an attractive area for ongoing and future research.

The paper is structured as follows. Section \ref{sec:background} discusses quality attributes and metrics in classical and quantum systems. Section \ref{sec:process} describes the mapping review methodology, followed by results in Section \ref{sec:results}. Finally, challenges are discussed in Section \ref{sec:challenges}, and conclusions are presented in Section \ref{sec:conclusions}.
