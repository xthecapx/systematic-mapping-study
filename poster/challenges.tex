Conducting a systematic mapping review in a rapidly evolving field presents methodological challenges. Despite quantum computing receiving significant investment from industry leaders such as IBM, Microsoft, and Google, the research domain exhibits limited evidence regarding quality management practices in hybrid architectures.

Moreover, \cite{40,26,Zhao_2021} emphasizes the lack of empirical studies on quality in quantum systems. Hence, in the upcoming years, developing observability tools \footnote{\textit{Characteristic of software and systems related to the information they generate that allows them to be monitored and understood more comprehensively, including at runtime.} \cite{Costa_Bachiega_Carvalho_Rosa_Araujo_2022}} and testing quality attributes in quantum systems represent a clear research opportunity.

Multiple researchers \cite{40,26,Zhao_2021} have identified a critical deficiency in empirical studies examining quality aspects of quantum systems. This gap creates substantial research opportunities, particularly in the development of observability tools \cite{Costa_Bachiega_Carvalho_Rosa_Araujo_2022} and the evaluation of quality attributes while working in quantum systems.

The current state of QC necessitates classical systems for crucial aspects of software development, including result preservation and business logic storage. This integration between quantum and classical components creates another substantial research opportunity: the development of monitoring tools for hybrid quantum-classical systems.

Furthermore, the field requires quantum-specific standards. In \cite{Gill2022} Sukhpal Singh Gill discusses various quantum technologies currently being used, including Trapped Ion, Superconducting, Silicon, Photonic, and Topological approaches. The wide range of hardware implementations of qubits complicates the establishment of software standards, since quantum algorithms may be executed differently by different providers and technologies. While some studies \cite{Piattini2020, 26, 40} address quality aspects in quantum systems, these guidelines remain general and lack empirical validation due to the early developmental stage of the field.